\chapter{Základní údaje o pohoří}
\label{Pohori}
%% -------------------------------------------------- %%
\def\figurename{Obr.} % Figure name
\def\tablename{Tab.} % Table name
\def\figureautorefname{obr.} % Autoreference 
\def\tableautorefname{tab.} % Autoreference
\def\chapterautorefname{kapitola} % Autoreference
%% -------------------------------------------------- %%

\section{Montblanský masiv}
Montblanský masiv je pohoří v Západních Alpách, který se nachází na území Francie, Itálie a Švýcarska. I když se rozkládá na území tří zemí, většina rozlohy přesněji 1\:500\:km2 tohoto pohoří se nachází na území Francie. Trojmezí těchto států leží poblíž vrcholu Mont Dolent (3\:820\:m\:n.\:m.). Montblanský masiv je výjimečné pohoří v rámci Alp a dokonce i celé Evropy, neboť se zde nachází nejvyšší hora v Alpách a na celém evropském kontinentu, a to Mont Blanc s výškou 4\:810\:m\:n.\:m. Hřeben tohoto horského masivu má délku 46\:km při maximální šířce kolem 20\:km. 

Na území Montblanského masivu se nachází rozmatiná krajina, která zahrnuje vysoké, impozantně zaledněné vrcholy, skalní stěný, které patří mezi nejvyšší v Evropě, skalní věže a jehly. Kromě toho lze také pozorovat horské hřebeny a ledovcová jezera a v neposlední řadě také trávou a horskými loukami porostlé vrcholky.

\section{Historie vzniku a geologie}
Pohoří Mont Blanc je tvořeno nejstaršími krystalickými horninami (zejména žulou a rulou), které zde existovaly v dobách, kdy jiné alpské masivy byly stále zatopeny oceánskou vodou. Jeho vznik (stejně jako v případě všech ostatních částí Alp) souvisí s tzv. Alpinským vrásněním, způsobeným nárazem Africké a Euroasijské litosférické desky. Tento horotvorný proces započal na konci druhohor (v křídě) před cca 145 miliony let, a pokračoval přes třetihory a čtvrthory až dodnes. Vyvrásnění oceánského dna vyzvednulo masiv do nejvyšších poloh a okolní sedimentární vrstvy byly sneseny. Na západním konci masivu, v oblati vrcholu Mont Joly (2\:525\:m), lze pozorovat rozsáhlé vápencové příkrovy, které překrývají základní krystalický materiál. Pevná žula vytváří charakteristické skalní útvary v oblastech jako Aiguilles de Chamonix a nebo Arete di DIable, které zahrnují ostré hřebeny a skalní jehly a věže.

\section{Ledovce}
Ledovci pokrývají většinu Montblanského masivu, rozkládají se zde na celkové ploše 170\:km2, přičemž většina z nich (110\:km2) se nachází na území francouzské části pohoří. Největším a nejspíše i nejslavnějším ledovcem v této oblasti je Mer de Glace, jehož celková délka dosahuje 7.5\:km (se započtením zdrojových ledovců v nejvyšších partiích masivu dokonce 11 km). Dalšími místními ledovci patří například Glacier d'Argentiére a Glacier des Bossons, které spadají do údolí řeky Arve v okolí Chamonix, a dále Glacier du Trient a Glacier de Saleina, které směřují naopak do Švýcarska. Na italské straně pohoří jsou ledovci obecně mnohem kratší, což je způsobeno strmějšími svahy ve srovnání s francouzskou stranou. Největším ledovcem na této straně je Ghiacciaio del Miage, který se táhne v délce asi 10\:km od sedla Col du Bionnassay, a stává se tak největším a nejdelším ledovcem v Itálii. Tento ledovec má zajímavou vlastnost, protože jeho povrch je téměř kompletně pokryt materiálem, který se uvolňuje z okolních svahů.

\section{Klima}
Hřeben Montblanského masivu nejen tvoří hranici tří států, ale zároveň rozděluje dva různé klimatické regiony. Klima v této oblasti je obecně vysokohorské a chladné. Srážky sem obvykle přináší západní proudění vzduchu, přičemž jejich množství se výrazně mění s nadmořskou výškou. V Chamonix (1\:030\:m\:n.\:m.) dosahuje průměrný roční úhrn srážek 1\:020\:mm, ale v sedle Col du Midi (3\:500\:m\:n.\:m.) už dosahuje 3\:100\:mm. Na vyšších nadmořských výškách opět klesá, takže oblast kolem vrcholu Mont Blancu (cca 4\:300\:m\:n.\:m.) má dlouhodobý roční srážkový úhrn okolo 1\:100\:mm. Leden je nejchladnějším měsícem s dlouhodobým průměrem teploty vzduchu -4.3\:°C a 116\:mm srážek, zatímco červenec je nejteplejším měsícem s průměrnou teplotou +15.6\:°C a 119\:mm srážek. Nejsušším měsícem, tedy měsícem s průměrně nejnižším srážkovým úhrnem, je duben (pozn. všechny údaje jsou pro oblast Chamonix).

\section{Ochrana území}
Ze zprávy uvedené na strankách UNESCO ze dne 8. června 2000: 

"Mont Blanc, jehož vrchol (4\:807\:m) je nejvyšším v Evropě, tvoří unikátní soubor ledovců a vysokohorských oblastí, které mají mimořádný přírodní i kulturní význam, neboť stály u zrodu horolezeckých sportů, zejména alpinismu.

Tento výjimečný krajinný ráz, vyznačující se vznešeností svých vrcholů a mocí svých ledovců, je skutečně jedním z mýtů horolezectví, na stejné úrovni jako Everest či Annapurna.

Je třeba ho oddělit od celku Vanoise - Grand Paradis, ke kterému nepatří, a zařadit ho na seznam samostatně."

Montblanský masiv je tedy na seznamu předněžných kandidátů.

Z další zprávy ze dne 30. ledna 2008 je zmíněno následující:

Masiv Mont-Blanc je neporušený a nedotčený a usiluje o to, aby lidská činnost v údolích byla slučitelná s ochranou hor. V roce 1991 Itálie, Francie a Švýcarsko založily Conférence Transfrontalière Espace Mont-Blanc (CTMB) za účasti Itálie, Francie a Švýcarska a za předsednictví francouzského ministra životního prostředí. Cílem CTMB je dosáhnout mezinárodní ochrany a posílení masivu Mont-Blanc. V roce 2005 vypracovala CTMB Plán udržitelného rozvoje, jehož cílem je ochrana a správa masivu a uplatňování zásad Alpské úmluvy při hospodářské činnosti v údolích. V roce 1991 Itálie, Francie a Švýcarsko založilo sdružení na ochranu životního prostředí Pro Mont-Blanc pro přeshraniční a mezinárodní ochranu masivu.


\section{Nejvyšší vrcholy}
V Montblanském masivu se nachází na 18 horských štítů převyšujících 4\:000\:m n.\:m (viz \autoref{Obr:tabulka_4000})


\begin{table}[h!]
    \begin{tabular}{|c|c|}
        \hline
        Název & výška nad mořem [m\:n.\:m.] \\ \hline
        Mont Blanc & 4\:810,45 \\ \hline
        Mont Blanc de Courmayeur & 4\:748 \\ \hline
        Rocher de la Tournette & 4\:677\\ \hline
        Grande Bosse & 4\:547 \\ \hline
        Petite Bosse & 4\:513 \\ \hline
        Mont Maudit & 4\:465 \\ \hline
        Pointe Louis Amédée & 4\:460 \\ \hline
        Aiguilles Belles Étoiles & 4\:354 \\ \hline
        Dôme du Goûter & 4\:304 \\ \hline
        Mont Blanc du Tacul & 4\:248 \\ \hline
        Grand Pilier d'Angle & 4\:243 \\ \hline
        Grandes Jorasses & 4\:208 \\ \hline
        Aiguille Verte& 4\:122 \\ \hline
        Aiguille Blanche & 4\:108 \\ \hline
        Aiguille de Bionnassay & 4\:052 \\ \hline
        Dent du Géant & 4\:013 \\ \hline
        Punta Baretti & 4\:013 \\ \hline
        Piton des Italiens & 4\:002 \\ \hline
    \end{tabular}
\caption[Tabulka 4OOO]{Tabulka horských štítů převyšujících 4\:000 m\:n.\:m.}
\label{Obr:tabulka_4000}
\end{table}

\section{Historie TMB}
V prvních dobách byly údolí kolem Mont Blancu osídlena malými zemědělskými komunitami. Zemědělci zde byli, aby využili bohatou zemědělskou půdu a měli málo potřeb, ani mnoho přání, cestovat hluboko do hor. Zemědělci cestovali do větších vesnic níže po údolí, aby prodali své plodiny. Obchodníci podnikali cestu údolími z větších měst, aby prodali své různé zboží zemědělcům. V některých případech by zemědělci a obchodníci přecházeli horské průsmyky, aby se dostali do sousedních údolí.

Pak přišla doba raných objevitelů a vědců. V 18. století byli vědci nadšení porozumět horám a jejich ledovcům. Kvůli své velikosti a kráse přitahoval Mont Blanc velké množství objevitelů a vědců. Kde to bylo možné, cestovali po stezkách používaných místními zemědělci a obchodníky a tyto propojovali, aby šli stále dál. V roce 1767 švýcarský aristokrat a fyzik Horace Benedict de Saussure vyrazil z údolí Chamonix se svým doprovodem na vědeckou expedici, která je obecně připisována jako první úplná cesta kolem Mont Blancu.

Ale v dobe viktoriánské éry se věci začaly zrychlovat. Dobrodružní Viktořiánci měli čas a peníze na cestování pro radost. Viděli krásné kresby a četli vzrušující vědecké zprávy publikované vědci, a tak chtěli přijít a zažít to na vlastní kůži. To byl velmi raný začátek horského turismu. Někteří místní zemědělci se obrátili k horskému vedení a vedli Viktořany, kteří často pohodlně seděli na zádech mezka, podél stezek, které zkoumal de Saussure.

Od té doby se horská turistika nadále rozvíjí a získává na popularitě. Dnes je Tour du Mont Blanc pravděpodobně jednou z nejpopulárnějších horských túr na světě. Trasa následuje části původní trasy de Saussurea, ale byla vyvinuta a přizpůsobena během let.

