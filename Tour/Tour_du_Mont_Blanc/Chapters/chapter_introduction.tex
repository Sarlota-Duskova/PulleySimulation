\chapter*{Úvod}
\label{Uvod}
%% -------------------------------------------------- %%
\def\figurename{Obr.} % Figure name
\def\tablename{Tab.} % Table name
\def\figureautorefname{obr.} % Autoreference 
\def\tableautorefname{tab.} % Autoreference
\def\chapterautorefname{kapitola} % Autoreference
%% -------------------------------------------------- %%
\addcontentsline{toc}{chapter}{Úvod}
Pro svoji praktickou část jsem si vybrala Tour du Mont Blanc (viz \autoref{Pohori}). V této kapitole jsem shrnula základní geografické a geologické údaje, jako je nejvyšší vrchol, rozloha, geologické složení a historie vzniku a na závěr informace o turistickém značení v dané oblasti. V kapitole Denní program přechodu (viz \autoref{Program_prechodu}) jsem uvedla výchozí a koncový bod, denní převýšení a délku denních etap s jejich časovým harmonogramem. Přiložila jsem i výškový profil každé denní etapy vygenerovaný pomocí mapy.cz. Uvedla jsem dostupnost vody, elektřiny a mobilního signálu v dané oblasti i možnost případného doplnění zásob. Dále jsem pro každou etapu zvolila vhodná nouzová sestupová místa. V kapitole Seznam potřebných věcí (viz \autoref{Potrebny_material}) jsem uvedla stručný seznam věcí potřebných na cestu. V poslední kapitole Rozpočet akce (viz \autoref{Rozpocet_akce}) jsem uvedla předpokládané finanční výdaje na realizaci tohoto přechodu. Rozpočet akce jsem koncipovala pro známou, a proto zde neuvádím náklady za průvodcovskou činnost. V rozpočtu jsou zahrnuty položky jako je doprava, ubytování, strava, pojištění, případně dokoupení potřebného materiálu na tuto túru. V závěrečné kapitole (viz \autoref{Souhrn}) jsem stručně shrnula túru, potřebný materiál a její rozpčet.