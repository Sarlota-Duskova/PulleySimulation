\chapter{Rozpočet akce}
\label{Rozpocet_akce}
%% -------------------------------------------------- %%
\def\figurename{Obr.} % Figure name
\def\tablename{Tab.} % Table name
\def\figureautorefname{obr.} % Autoreference 
\def\tableautorefname{tab.} % Autoreference
\def\chapterautorefname{kapitola} % Autoreference
%% -------------------------------------------------- %%

Pokud budu uvažovat nejlevnější variantu tak to samozřejmě bude varianta na těžko, tedy se spacákem, karimatkou a případně stanem. Pokud budu uvažovat variantu na lehko a tedy s přespáním na chatách a polopenzí dostaneme se na 555,4\:€. Cesta autem je z Prahy 975\:km, tam a zpět tedy 1950\:km, pokud budu počítat spotřebu 3,0\:Kč/km vyjde cesta za naftu na 5\:850\:Kč. Dále se musí započítat Švýcarská dálniční známka, která stojí 40\:€. V neposlední řadě bude nutné přibalit nějaké jídlo na odpolední svačiny jako je protejnový mix a nebo protejnové sušenky cca 500\:Kč. Samozřejmě cestou potkáme spoustu chat a byla by zde tedy i možnost využít k obědu chatu. Cena by se tím, ale navýšila. Ve Švýcarsku je možné platit eurem a tedy ceny z CHF platí i pro eura. Cenu eura budu počítat 25\:Kč, vyjde ubytování s polopenzí na 13\:885\:Kč, dálniční známka na 1\:000\:Kč. Celkem to tedy dělá 17\:810\:Kč.

U varianty na lehko by cesta za cenu a dálniční známku zůstala stejná. Adventure menu vychází na cca 190\:Kč, maximálně bychom mohli vzít 7\:ks to je 1\:330\:Kč. Na snídani kaše na 10 dní cca 300\:Kč. Přes den protejnový mix a nebo protejnové sušenky cca 500\:Kč. Na poslední 4 dny by bylo nutné jídlo nakoupit a tedy cena cca 1\:000. Jídlo celkem vychází na 3\:130\:Kč. Celkem to dělá 6\:555\:Kč.