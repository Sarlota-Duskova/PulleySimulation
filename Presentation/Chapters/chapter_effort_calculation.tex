\chapter{Výpočty}
\label{Vypocty}
%% -------------------------------------------------- %%
\def\figurename{Obr.} % Figure name
\def\tablename{Tab.} % Table name
\def\figureautorefname{obr.} % Autoreference 
\def\tableautorefname{tab.} % Autoreference
\def\chapterautorefname{kapitola} % Autoreference
%% -------------------------------------------------- %%

\section{Výpočet síly potřebné k vytažení zachraňovaného lezce}
\subsection{Ideální kladkostroj}
V této kapitole uvedu výpočet, který je zakomponován ve webové aplikaci. Pro výpočet se 100\% účinností, tedy ideálním kladkostrojem uvažuji ve webové aplikaci v případě pokud uživatel nezadá konkrétní účinnost kladky.

\noindent Obecná rovnice síly:
%% -------------------------------------------------- %%
%% -------------------- Equation --------------------- %%
%% -------------------------------------------------- %%
\begin{equation}
    \label{eqn:basic_forces}
    S = \frac{F}{\mu \cdot n}
\end{equation}
%% -------------------------------------------------- %%
%% -------------------- Equation --------------------- %%
%% -------------------------------------------------- %%
\begin{equation}
    \label{eqn:basic_forces_2}
    S = \frac{m \cdot g}{\mu \cdot n}
\end{equation}

\begin{tabular}{l l c p{9.75cm}}
    kde: \hspace{0.25cm} & $S$ & -- & síla úsilí [N]\\
    \hspace{0.25cm} & $F$ & -- & síla zátěže [N]\\
    \hspace{0.25cm} & $m$ & -- & hmotnost zátěže [kg]\\
    \hspace{0.25cm} & $g$ & -- & gravitační konstanta $[9,80665\,m \cdot s^{-2}]$\\
    \hspace{0.25cm} & $\mu$ & -- & účinnost (rovné jedné pro ideální systém bez tření, zlomek menší než jedna u systémů v reálném světě se ztrátami energie v důsledku tření)\\
    \hspace{0.25cm} & $n$ & -- & mechanická účinnost systému\\
\end{tabular}
\\
\subsection{Kladkostroj 3:1}
Pokud uživatel zadá konkrétní účinnost kladky, uvažuji pro \textbf{systém 3:1} následující vzorec:
%% -------------------------------------------------- %%
%% -------------------- Equation --------------------- %%
%% -------------------------------------------------- %%
\begin{equation}
    \label{eqn:calculation_3_1}
    F = \frac{m}{1 + \frac{f_1}{100} + (\frac{f_1}{100} \cdot \frac{f_2}{100})}
\end{equation}
\subsection{Kladkostroj 5:1}
%% -------------------------------------------------- %%
%% -------------------- Equation --------------------- %%
%% -------------------------------------------------- %%
Uvažuji rozložení sil (viz~\autoref{Obr:5:1_forces_distribution}). Od pramene, kterým se vytahuje zachraňovaný se jednotlivé části lana mezi kladkami rozdělí na ${F_1}$ - ${F_8}$. Kladky označuji ${f_1}$ - ${f_4}$ a začínají s číslováním od zachraňovaného. Pro \textbf{systém 5:1} uvažuji následující vzorce:

%% -------------------------------------------------- %%
%% -------------------- Equation --------------------- %%
%% -------------------------------------------------- %%
\begin{equation}
    \label{eqn:1_calculation_5_1}
    F_2 = 1 - \frac{f_4}{100}
\end{equation}
%% -------------------------------------------------- %%
%% -------------------- Equation --------------------- %%
%% -------------------------------------------------- %%
\begin{equation}
    \label{eqn:2_calculation_5_1}
    F_3 = 2 - \frac{f_4}{100}
\end{equation}
%% -------------------------------------------------- %%
%% -------------------- Equation --------------------- %%
%% -------------------------------------------------- %%
\begin{equation}
    \label{eqn:3_calculation_5_1}
    F_4 = \frac{f_2}{100} \cdot \frac{f_3}{100}
\end{equation}
%% -------------------------------------------------- %%
%% -------------------- Equation --------------------- %%
%% -------------------------------------------------- %%
\begin{equation}
    \label{eqn:4_calculation_5_1}
    F_5 = 1 - (\frac{f_4}{100} \cdot \frac{f_2}{100})
\end{equation}
%% -------------------------------------------------- %%
%% -------------------- Equation --------------------- %%
%% -------------------------------------------------- %%
\begin{equation}
    \label{eqn:5_calculation_5_1}
    F_6 = 2 - (\frac{f_4}{100} \cdot \frac{f_2}{100})
\end{equation}
%% -------------------------------------------------- %%
%% -------------------- Equation --------------------- %%
%% -------------------------------------------------- %%
\begin{equation}
    \label{eqn:6_calculation_5_1}
    F_7 = \frac{f_5}{100} \cdot \frac{f_1}{100}
\end{equation}

\noindent Po dosazení do následující rovnice:
%% -------------------------------------------------- %%
%% -------------------- Equation --------------------- %%
%% -------------------------------------------------- %%
\begin{equation}
    \label{eqn:7_calculation_5_1}
    F_{celk} = F_3 + F_6 + F_7
\end{equation}

\noindent Získáme následující rovnici:
%% -------------------------------------------------- %%
%% -------------------- Equation --------------------- %%
%% -------------------------------------------------- %%
\begin{equation}
    \label{eqn:8_calculation_5_1}
    F_{celk} = (2 - \frac{f_4}{100}) + (2 - (\frac{f_4}{100} \cdot \frac{f_2}{100})) + (\frac{f_5}{100} \cdot \frac{f_1}{100})
\end{equation}

\noindent Po dosazení a zkrácení dostávám následující rovnici, se kterou počítám ve webové aplikaci.
%% -------------------------------------------------- %%
%% -------------------- Equation --------------------- %%
%% -------------------------------------------------- %%
\begin{equation}
    \label{eqn:9_calculation_5_1}
    F = \frac{m}{F_{celk}}
\end{equation}

%% -------------------------------------------------- %%
%% -------------------- Equation --------------------- %%
%% -------------------------------------------------- %%
\begin{equation}
    \label{eqn:10_calculation_5_1}
    F = \frac{m}{4 + \frac{f_1}{100} - \frac{f_4}{100} + (\frac{f_2}{100} \cdot \frac{f_3}{100}) \cdot (\frac{f_4}{100} - 1 - \frac{f_1}{100} + \frac{f_1}{100} \cdot \frac{f_4}{100})}
\end{equation}
\subsection{Kladkostroj 7:1}
Uvažuji rozložení sil (viz~\autoref{Obr:7:1_forces_distribution}). Od pramene, kterým se vytahuje zachraňovaný se jednotlivé části lana mezi kladkami rozdělí na ${F_1}$ - ${F_8}$. Kladky označuji ${f_1}$ - ${f_3}$ a začínají s číslováním od zachraňovaného. Pro \textbf{systém 7:1} uvažuji následující vzorce:

%% -------------------------------------------------- %%
%% -------------------- Equation --------------------- %%
%% -------------------------------------------------- %%
\begin{equation}
    \label{eqn:1_calculation_7_1}
    F_2 = 2 - \frac{f_3}{100}
\end{equation}
%% -------------------------------------------------- %%
%% -------------------- Equation --------------------- %%
%% -------------------------------------------------- %%
\begin{equation}
    \label{eqn:2_calculation_7_1}
    F_3 = 4 - (\frac{f_3}{100}^2 + \frac{f_2}{100})
\end{equation}
%% -------------------------------------------------- %%
%% -------------------- Equation --------------------- %%
%% -------------------------------------------------- %%
\begin{equation}
    \label{eqn:3_calculation_7_1}
    F_4 = F_3 - F_2
\end{equation}
%% -------------------------------------------------- %%
%% -------------------- Equation --------------------- %%
%% -------------------------------------------------- %%
\begin{equation}
    \label{eqn:4_calculation_7_1}
    F_5 = \frac{f_3}{100}
\end{equation}
%% -------------------------------------------------- %%
%% -------------------- Equation --------------------- %%
%% -------------------------------------------------- %%
\begin{equation}
    \label{eqn:5_calculation_7_1}
    F_6 = F_4 + F_5
\end{equation}
%% -------------------------------------------------- %%
%% -------------------- Equation --------------------- %%
%% -------------------------------------------------- %%
\begin{equation}
    \label{eqn:6_calculation_7_1}
    F_7 = F_6 \cdot \frac{f_1}{100}
\end{equation}

\noindent Po dosazení do následující rovnice:
%% -------------------------------------------------- %%
%% -------------------- Equation --------------------- %%
%% -------------------------------------------------- %%
\begin{equation}
    \label{eqn:7_calculation_7_1}
    F_{celk} = F_3 + F_7
\end{equation}

\noindent Získáme následující rovnici:
%% -------------------------------------------------- %%
%% -------------------- Equation --------------------- %%
%% -------------------------------------------------- %%
\begin{equation}
    \label{eqn:8_calculation_7_1}
    F_{celk} = (4 - (\frac{f_3}{100}^2 + \frac{f_2}{100})) + (F_6 \cdot \frac{f_1}{100})
\end{equation}

\noindent Po dosazení a zkrácení dostávám následující rovnici, se kterou počítám ve webové aplikaci.
%% -------------------------------------------------- %%
%% -------------------- Equation --------------------- %%
%% -------------------------------------------------- %%
\begin{equation}
    \label{eqn:9_calculation_7_1}
    F = \frac{m}{F_{celk}}
\end{equation}

%% -------------------------------------------------- %%
%% -------------------- Equation --------------------- %%
%% -------------------------------------------------- %%
\begin{equation}
    \label{eqn:10_calculation_7_1}
    F = \frac{m}{4 - \frac{f_2}{100} - \frac{f_3}{100}^{2} + 3 \cdot \frac{f_1}{100} - \frac{f_1}{100} \cdot \frac{f_3}{100}^{2} - \frac{f_2}{100} \cdot \frac{f_1}{100}}
\end{equation}

\begin{tabular}{l l c p{9.75cm}}
    kde: \hspace{0.25cm} & $F$ & -- & vytahovaná hmotnost zachraňovaného lezce [kg]\\
    \hspace{0.25cm} & $m$ & -- & hmotnost vytahovaného lezce [kg]\\
    \hspace{0.25cm} & $F_1$ - $F_8$ & -- & mezivýpočet účinnosti systému [\%]\\
    \hspace{0.25cm} & $f_1$ - $f_4$ & -- & účinnost kladkostroje [\%]\\
\end{tabular}
\\
\section{Výpočet vytažené délky lana k zachránění lezce o požadovanou vzdálenost}
Při výpočtu vycházím z již vypočtených účinností systému $F_{celk}$.
%% -------------------------------------------------- %%
%% -------------------- Equation --------------------- %%
%% -------------------------------------------------- %%
\begin{equation}
    \label{eqn:calculation_distance}
    l = l_{poz} \cdot F_{celk}
\end{equation}

\begin{tabular}{l l c p{9.75cm}}
    kde: \hspace{0.25cm} & $l$ & -- & skutečná vytažená délka lana k zachraně lezce o požadovanou vzdálenost [m]\\
    \hspace{0.25cm} & $l_{poz}$ & -- & požadovaná vzdálenost k vytažení zachraňovaného lezce [m]\\
    \hspace{0.25cm} & $F_{celk}$ & -- & celkový výpočet účinnosti systému [\%]\\
\end{tabular}