\chapter*{Úvod}
\label{Uvod}
%% -------------------------------------------------- %%
\def\figurename{Obr.} % Figure name
\def\tablename{Tab.} % Table name
\def\figureautorefname{obr.} % Autoreference 
\def\tableautorefname{tab.} % Autoreference
\def\chapterautorefname{kapitola} % Autoreference
%% -------------------------------------------------- %%
\addcontentsline{toc}{chapter}{Úvod}
Budu se zabývat problematikou záchrany ve skalním terénu. Uvedu základní potřebný materiál k záchraně (viz \autoref{Potrebny_material}). Zaměřím se na základní záchranné techniky klientů ve volném terénu či na zajištěných cestách (viz \autoref{Mozne_zpusoby_zachrany}). Výstupem bude vytvoření webové aplikace (viz \autoref{Webova_aplikace}), kde si uživatel bude moct nastavit hmotnost zachraňovaného lezce, zavést do výpočtu různé případy účinnosti kladek použitých v systému a umožní vypočítat skutečnou vytaženou délku lana k záchraně lezce o požadovanou vzdálenost. Všechny výpočty ve webové aplikaci budou shrnuty v následující kapitole (viz \autoref{Vypocty}). Cílem této aplikace bude umožnit uživateli jednoduše nasimulovat situaci a vypočítat jakou hmotnost bude muset vytáhnout, pokud jeho spolulezec či klient nebude moct překonat určitou pasáž, či se zraní a nebude moct pokračovat ve výstupu.