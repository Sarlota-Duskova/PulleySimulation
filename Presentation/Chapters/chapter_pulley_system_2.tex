\subsection{Systém vytažení protiváhou "Straus"}
%% -------------------------------------------------- %%
\def\figurename{Obr.} % Figure name
\def\tablename{Tab.} % Table name
\def\figureautorefname{obr.} % Autoreference 
\def\tableautorefname{tab.} % Autoreference
\def\chapterautorefname{kapitola} % Autoreference
%% -------------------------------------------------- %%
Nejjednodušší dopomoc či vytažení vlastní váhou.
\\
\\
\textbf{Postup:}

1. Kravskou smyčkou zajistíme jístitko či půllodní uzel. 

2. Umístímě trojitý 1,5\,m prusík a zafixujeme jej ve stanovišti.

3. Zrušíme zablokované jištění a pouze lano procvakneme ve stanovišti karabinou.

4. Na volném konci lana použijeme otevřenou gardu umístěnou v centrálním oku sedacího úvazku.

5. Zatížením volného konce vlastní vahou a současným přitažením lana jdoucímu k lezci jej takto vytáhneme. Nutné vždy posunout prusík, který nám zajistí vytažené lano.
\\
\\
\textbf{Nejčastější chyby:}

1. Špatné uvázání prusíku.

2. Přeskočení prusíku v karabině, lze vyřešit umístěním reversa před prusík a zkrácením prusíku na nejmenší délku či použití tiblocu nebo microtraxtionu.

3. Neprodloužení sebejištění a tedy špatné zatěžování \cite{climbing_school_2022}.

\subsection{Expresflaschenzug}

Metoda, která spíše slouží jako pomoc druholezci v situacích, kdy se v cestě nachází těžší místo, které nemůže lezec svépomocí překonat. 
\\
\\
\textbf{Postup:}

1. Pokud dobíráme spolulezce přes reverso či půllodní uzel, zajístíme jíštění kravskou smyčkou. 

2. Na zatížené lano umístíme prusík 1,5\,m, který zkrátíme na nejmenší možnou délku a procvakneme jím karabinu.

3. Volný pramen lana protáhneme karabinou s prusíkem. Uvolníme zablokované jistítko a dobíráme lezce \cite{climbing_school_2022}.

\subsection{Seirollflaschenzug}

Nejjednodušší sestrojení kladkostroje. Technika je stejná jako u předešlé techniky Expresflaschenzug, pouze nyní zrušíme jištění a pouze lano procvakneme do karabiny. Jedná se o kladkostroj s účinností 3:1.
\\
\\
\textbf{Postup:}

1. Lezec nám visí v prusíku umístěném za půllodním uzlem. Kravskou smyčkou zajistíme půllodní uzel. 

2. Na zatížené lano umístíme prusík 1,5\,m, který zkrátíme na nejmenší možnou délku a procvakneme jím karabinu.

3. Volný pramen lana protáhneme karabinou s prusíkem. Uvolníme zablokované jistítko a zrušíme jej, lezec nám visí v pomocném prusíku. Lano z jistítka procvakneme karabinou do jistícího stanoviště \cite{climbing_school_2022}. 

\subsection{Flaschenzug}

Kladkostroj s účinností 5:1.
\\
\\
\textbf{Postup:}

1. Lezec nám visí v prusíku umístěném za půllodním uzlem. Kravskou smyčkou zajistíme půllodní uzel. 

2. Na zatížené lano umístíme prusík 1,5\,m, který zkrátíme na nejmenší možnou délku a procvkaneme jím karabinu.

3. Do jistícího stanoviště umístíme 5,0\, prusík, který provážeme skrz karabinu a protáhneme skrz karabinu s prusíkem. 

4. Karabinou protáhneme volný konec lana \cite{climbing_school_2022}.

\subsection{Kladkostroj 7:1}

Kladkostroj s účinností 7:1. 
\\
\\
\textbf{Postup:}

1. Lezec nám visí v prusíku umístěném za půllodním uzlem. Kravskou smyčkou zajistíme půllodní uzel. 

2. Na zatížené lano umístíme prusík 1,5\,m, který zkrátíme na nejmenší možnou délku a procvakneme jím karabinu.

3. Za půllodním uzlem, tedy volným koncem lana umístíme 5,0\,m prusík, který protáhneme skrz karabinu s 1,5\,m prusíkem umístěném na zatíženém laně. 

4. Do 5,0\,m prusíku procvakneme karabinu a volný konec lana.
