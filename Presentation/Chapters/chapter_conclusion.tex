\chapter*{Závěr}
\addcontentsline{toc}{chapter}{Závěr}
\fancyhead{}
%% -------------------------------------------------- %%
\def\figurename{Obr.} % Figure name
\def\tablename{Tab.} % Table name
\def\figureautorefname{obr.} % Autoreference 
\def\tableautorefname{tab.} % Autoreference
\def\chapterautorefname{kapitola} % Autoreference
%% -------------------------------------------------- %%

V úvodu práce byl popsán základní materiál potřebný k záchraně spolulezce či klienta. Dále jsem se zaměřila na základní záchranné techniky klientů ve volném terénu či na zajištěných cestách. Popsala jsem kladkový efekt a uvedla kladkostroje se systém 1:1, 2:1, 3:1, 4:1, 5:1, 7:1. Stručně jsem popsala provedení nejčastěji používaných kladkostrojů. 

Výstupem této práce bylo vytvořit webovou aplikaci, kde si uživatel bude moct nastavit hmotnost zachraňovaného lezce, zavést do výpočtu různé případy účinnosti kladek použitých v systému a umožní vypočítat skutečnou vytaženou délku lana k záchraně lezce o požadovanou vzdálenost. Všechny výpočty ve webové aplikaci byly shrnuty v kapitole zaměřené na výpočet.

Cílem této aplikace je umožnit uživateli jednoduše nasimulovat situaci a vypočítat jakou hmotnost bude muset vytáhnout, pokud jeho spolulezec či klient nebude moct překonat určitou pasáž, či se zraní a nebude moct pokračovat ve výstupu. 

Samozřejmě záleží na mnoha dalších vlivech, které v aplikaci nejsou uvažovány. Výsledek je tedy pouze orientační. V realném případě bude mít vliv na výslednou sílu tloušťka lana, protažení prusíku, vlhkost lana či opotřebení pomůcek zakomponovaných v systému. Také bude mít vliv na výslednou sílu úhel, pod kterým bude lano vytahováno, to samozřejmě bude záviset na pracovním prostoru. 